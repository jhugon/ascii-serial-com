\documentclass{customdocclass}

\title{ASCII-Serial-Com Data Format}

\begin{document}

\maketitle

This is the format for a serial communication protocol based on ASCII. The
protocol is not meant to be efficient, but to be easily human readable.

\textbf{This protocol defines a host and a device (like USB). Command types
have different meanings depending on if they are sent from a host or device}

\section{Data Frame}

Data frames are made up of 8-bit bytes. Only the following symbols are present in frames:

\vspace{1em}

a-zA-Z.> newline (\textbackslash n) and space

\vspace{1em}

All of the characters are 8-bit ASCII (with the MSB set to 0).

The maximum frame size is 64 bytes to reduce buffer size in microcontrollers.

\subsection{Data Frame Format}

The data frame consists of the following:

\begin{itemize}
  \item The start of frame character `>'
  \item Two bytes of version and format information represented in the characters a-zA-Z0-9
  \item One byte of command and/or message type as lower case letters a-z
  \item Data is then sent as 0-9A-F (the letters must be capital)
  \begin{itemize}
    \item The space character may be used to group data
  \end{itemize}
  \item The end of data character `.'
  \item A CRC code represented as ASCII hex 0-9A-F (capitals)
  \item The end of frame character `\textbackslash n'
\end{itemize}

\section{Command and/or Message Types}

\begin{center}
\small
\begin{tabularx}{\textwidth}{|X|l|X|X|} \hline
\multicolumn{2}{|c|}{Command/Type} & & \\ \hline
Description & Code & Host Sent Data & Dev Sent Data \\ \hline
Read register & r & Reg num & Reg num space reg content \\ \hline
Write register & w & Reg num space reg content & Reg num (for success) \\ \hline
\end{tabularx}
\end{center}

\end{document}
