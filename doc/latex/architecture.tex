\documentclass{customdocclass}

\title{ASCII-Serial-Com Software Architecture}
\author{Justin Hugon}
\date{2020-06-13}

\begin{document}

\maketitle

This document describes the high-level architecture for a serial communication
protocol based on ASCII.

\section{Overview}

Assuming UART receives data into a buffer and data in another buffer is
transmitted. ASCII-Serial-Com is part of a data processing loop (likely low
priority) on a microcontroller.

For just the basic register control interface (not streaming), the device could
just have a function in it's event loop like:

\begin{lstlisting}[language=C]
void pollAsciiSerialCom(uint8_t* inBuffer, uint8_t inBufSize, uint8_t* outBuffer, uint8_t outBufSize, void* registers, uint16_t regSize);
\end{lstlisting}

That would then call functions below to read from inbuffer, frame, check,
unpack, and dispatch functions to the appropriate function for action. The
outBuffer is there for responses, and registers are there for reads and writes.

A macro would define the bit-width of each register.

May want to have an argument that is the register to store the number of
checksum errors in.

\section{Receiver}

\begin{itemize}
  \item Framer: identifies and returns frames out of the incoming serial byte stream
  \begin{itemize}
    \item Inputs: byte stream
    \item Outputs: data frame
  \end{itemize}
  \item Checker: computes and verifies the checksum of the frame
  \begin{itemize}
    \item Inputs: data frame
    \item Outputs: valid flag
  \end{itemize}
  \item Unpacker: unpacks the message type (command) and data from the frame
  \begin{itemize}
    \item Inputs: data frame
    \item Outputs: message type, data
  \end{itemize}
  \item Dispatcher (optional): sends different message types to different functions or threads
  \begin{itemize}
    \item Inputs: message type, data
  \end{itemize}
\end{itemize}

\section{Transmitter}

\begin{itemize}
  \item Packer: packs the message type (command) and data into a frame
  \begin{itemize}
    \item Inputs: command, data
    \item Outputs: data frame without checksum or end of frame
  \end{itemize}
  \item Checker: computes and verifies the checksum of the frame (can be part of packer)
  \begin{itemize}
    \item Inputs: data frame without checksum or end of frame
    \item Outputs: data frame
  \end{itemize}
  \item Streamer: puts the frame into the message stream
\end{itemize}

\section{Shared Components}

\begin{itemize}
  \item Checksum calculator
  \item Message type validation
  \item Data validation for message type (for python implementation)
\end{itemize}

\end{document}
